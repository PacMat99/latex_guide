% !TEX encoding = UTF-8
\documentclass[a4paper, 12pt]{article}
\usepackage[T1]{fontenc}
\usepackage[english,italian]{babel}

\usepackage{kantlipsum}

\newenvironment{citazione}[1]
{\begin{quotation}\begin{otherlanguage*}{#1}\small\itshape}
{\end{otherlanguage*}\end{quotation}}

\newenvironment{incipit}[1]
{\begin{citazione}{#1}\normalfont}
{\end{citazione}}

\begin{document}

\title{Il mio primo documento LaTeX}
\author{Mattia}
\date{\today}
\maketitle

Questo è il mio primo documento in \LaTeX{}.
Questo è un \textit{capoverso}, il primo capoverso di questo documento.
Se tutto va per il verso giusto le righe successive alla prima dovrebbero essere spostate leggermente più a sinistra perché non hanno il rientro iniziale.

\emph{Vediamo un po' cosa riusciamo a fare!}
Effettivamente tutto sembra funzionare.
Il rientro iniziale ha origini centinaia di anni fa ed è uno stile tipografico a cui non si è ancora trovata un'alternativa altrettanto valida.

\small Questo \`{e} un paragrafo con ``caratteri speciali''.

Ora proviamo a scrivere alcuni esempi di framebox:

\framebox[10cm][l]{framebox con testo a sinistra}

\framebox[10cm][r]{framebox con testo a destra}

\framebox[10cm][s]{framebox con testo distribuito}

{\scshape Questo è un Esempio di Gruppo in Maiuscoletto.}

\textsc{Questo è un Testo in Maiuscoletto scritto in modo Corretto.}

Facciamo un esempio di citazione con corpo leggermente più piccolo di un testo inglese scritto in corsivo utilizzando un gruppo:
\begin{quotation}
    \small\itshape 
    \begin{otherlanguage*}{english}
        Whether I shall turn out to be the hero
        of my own life, or whether that station will be
        held by anybody else, these pages must show.
    \end{otherlanguage*}
\end{quotation}
\`{E} un perfetto esempio dello stile di Dickens.

Ora vediamo un altro esempio.

\kant[1][1-4]

\begin{incipit}{english}
    \kant[2][1-4]
\end{incipit}

\kant[3][1-4]

\begin{citazione}{english}
    \kant[4][1-2]
\end{citazione}

\end{document}
